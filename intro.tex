%%%%%%%%%%%%%%%%%%%%%%%%%%%%%%%%
% intro.tex: Introduction to the thesis
%%%%%%%%%%%%%%%%%%%%%%%%%%%%%%%%
\chapter{Introduction}
\label{chap:intro}
\epigraph{\textit{The clues to galaxy formation and evolution are hidden in the fine details of galaxy structure.}}{\textit{Peng et al., 2002}}

The processes which govern the formation, growth, and eventual death of galaxies are uniquely difficult to investigate. A galaxy cannot ever be directly observed from its birth to its death; the only data available is a single snapshot of the Universe as it exists right now, in its current cosmological state. To begin to map out the complete evolutionary history of a galaxy, astronomers must instead use other clever, indirect methods.

Morphology is one of the most powerful tools for revealing the physical processes that shape the evolution of galaxies. Details of a galaxy's structure are known to be linked with its color \citep{Tully1982,Strateva2001,Baldry2004}, recent star-formation \citep{Conselice2006,Martin2007,Mignoli2009}, merger rate \citep{Hammer2009,Oesch2010,Smethurst2017}, and black hole activity \citep{Athanassoula1992,Friedli1993,Schawinski2010}, among others. There is no debate today that morphology is strongly linked to galactic evolution, but the extent to which these relationships hold is still difficult to quantify. Morphological classifications on scales large enough for results to claim statistical significance have been, in the past, unavailable. While expert visual classifications succeeded in accuracy, they lacked in numbers, and the opposite has been true for computational methods. 

The research described in this thesis examines the link between morphology and evolution using data from the Galaxy Zoo project, which uses crowd-sourcing to provide a ``best-of-both-worlds'' approach to morphological classifications. To date, over one million volunteers have identified the structures of over one million galaxies, providing the benefits of both visual inspection and large numbers. With these data, the ways in which morphology drives (or is driven by) a galaxy's evolution has been investigated on a scale previously unachievable. Three topics will be considered in detail: the influence of bars on AGN activity (Chapter~\ref{chap:baragn}), the dependence of observed wavelength on tracing different stellar populations (Chapter~\ref{chap:ukidss}), and the interplay between quenching mechanisms and morphological transformations of galaxies from redshift $z\sim1$ (Chapter~\ref{chap:gzh_red_disks}). This thesis also includes a detailed summary of the methodology used in collecting and reducing crowd-sourced data from Galaxy Zoo in the local Universe (Chapter~\ref{chap:methodology}) and introduces a new technique for debiasing high-redshift GZ classifications using data from simulated galaxies (Chapter~\ref{chap:ferengi}). First, this Introduction will describe the typical components of a galaxy resulting from its formation, then give a brief summary of morphological types as have been defined historically, as well as the current evidence linking morphology to galaxies' past histories. 
 

\section{Galaxy Formation}

\begin{figure}
\centering
\includegraphics[width=\textwidth,trim={.5cm .3cm 0cm .25cm},clip]{figures/galaxy_diagram.pdf}
\label{fig:galaxyfig}
\caption{A side-on (left) and face-on (right) view of typical spiral galaxies. The edge-on view gives a clear visual of the disk, bulge, and galactic halo components. The face-on view reveals the detailed spiral structure within the disk. \textbf{Left:} Hubble image of Sombrero galaxy, M104. Credit: \href{http://www.esa.int/spaceinimages/Images/2008/06/A_galaxy_and_its_halo}{European Space Agency}. \textbf{Right:} Hubble image of Pinwheel galaxy, M101. Credit: \href{http://sci.esa.int/hubble/38853-pinwheel-galaxy/}{European Space Agency}.  }
\end{figure}

Galaxies are believed to have initially formed from primordial density fluctuations shortly after the Big Bang, as predicted by the $\Lambda$CDM model of Cosmology \citep{Peebles1994,Ryden2006,Conselice2012,Silk2013}. In this model, fluctuations $\delta$ can be characterized as the contrast of the local density in a region $\rho$ as compared to the mean density of the Universe, $\bar{\rho}$: $\delta = \frac{\rho - \bar{\rho}}{\bar{\rho}}$. Gravitational instabilities will cause even small perturbations ($\delta<<1$) to grow exponentially with time; as $\delta \rightarrow 1$, the region collapses and breaks off from the expanding universe, becoming a self-gravitating structure.

The collapse of these protogalactic regions of overdensity triggered the formation of the first stars and galaxies around $z\sim11$, about 400 years after the big bang. At this time in the early Universe's history, the only baryonic matter in existence were Hydrogen and Helium. The very first stars to form in the first proto-galaxies were thus comprised of only these two elements; for this reason they are classified as ``extemely metal-poor stars'' (EMPs), or Population III stars. Heavier elements formed later in the cores of EMPs, allowing for the formation of the metal-rich stars more commonly observed in galaxies today. Different populations of stars are often classified by their metallicity, measured by the amount of heavier elements they contain relative to Hydrogen [Z/H]. Population II stars have metallcities of $\rm [Z/H] < 0.1 \%$; while they are still considered metal-poor, the presence of any metal requires that they were formed from gas generated from the deaths of the earlier Population III stars. Population I stars are the youngest observed, with metallicities $\rm [Z/H] \sim 2-3\%$. 

The structure and components of a typical galaxy is shown in Figure~\ref{fig:galaxyfig}. Most galaxies are believed to form with a disk component, which is a product of the dynamics governing the initial galaxy formation. As large gas clouds cool and collapse, conservation of angular momentum causes the cloud to flatten and increase rotation speed, resulting in a disk shape. Young, Population I stars tend to form in the spiral arms of the disk, and particularly dense regions of star formation can give the arms patchy appearances. Most galaxies are believed to contain a supermassive black hole in the center; while not observed directly, their influence on the galaxy in the form of feedback has been intensely studied (see Chapter~\ref{chap:baragn} for details). Surrounding the black hole and often taking up a significant part of the galaxy is the central bulge, which tends to be comprised of older Population II stars. The disk and distinction of the central bulge can be destroyed as the galaxy evolves, transforming the galaxy's morphology into an elliptical structure. It is believed that this morphological evolution is tied to the shutting-down, or quenching, of star-formation, due to ellipticals tending to host only older stellar populations (see Chapter~\ref{chap:gzh_red_disks}. Permeating the galaxy is the largest observable component: the galactic halo, which contains gas which fuels ongoing star-formation, and stars which extend to the outer regions of the galaxy. Last, all of the luminous components are embedded in a dark matter halo, which cannot be observed directly, but interacts with they baryonic matter gravitationally. 

While the preceding described the elements of a typical galaxy, this does not begin to describe the wide range of morphological features that can be found. Disk galaxies for instance may have any number of spiral arms or no arms at all, or show evidence of rings or bars. Many galaxies do not even exhibit disk or elliptical structure to begin with; but have irregular or clumpy appearances. This next section will describe in more detail the variances in the shapes of galaxies, and how the different structures are grouped into morphological classifications. 

\section{Morphological Categorization of Galaxies}

The oldest and most well-known system which categorizes galaxies based on their structure was developed by Edwin Hubble, commonly known as the ``Hubble Tuning-Fork'' \citep{Hubble1926}. Using a small sample of photometric images of nearby galaxies, Hubble identified two fundamental morphological classes: spirals, which exhibited well-defined disk structure and clear spiral arms, and ellipticals, whose light distributions were smoothed over a roughly spherical shape. Only 3\% of the sample had structures which deviated from these two categories, showing no evidence of rotational symmetry about a dominating nucleus; these were grouped together and labeled ``Irregular''. Although Hubble's system was originally based on a mere 400 galaxies, the classifications are still valid for describing the morphologies of the millions of galaxies identifiable today (albeit with some modications, ie. DeVaucouleur's revised system \citep{DeVaucouleurs1963}).

An example of Hubble's Tuning Fork is shown in Figure \ref{fig:tuningfork}. The classifications defined on the Tuning Fork are as follows:

\begin{figure}
\centering
\includegraphics[width=\textwidth]{figures/TuningForkKaren.jpg}
\label{fig:tuningfork}
\caption{The Hubble Tuning fork with gri-composite SDSS images as examples of the various types. Credit: Karen Masters and The Sloan Digital Sky Survey (SDSS) Collaboration.}
\end{figure}

\subsection{Ellipticals}

The left side of the tuning fork contains elliptical galaxies, labeled ``E''. These were originally identified as circular through flattened ellipses whose luminosity faded smoothly from the center to ``indefinite edges.'' The only other structural feature evident to subdivide this class were their ellipticities, defined in the traditional way $e=(a-b)/a$. A number is added to the label that represents the ellipticity, with the decimal omitted, whereby E0 would represent a purely spherical elliptical ($e=0$), and E7 being the most elongated ($e=0.7$). Hubble assumed that any galaxy with an ellipticity higher than 0.7 was no longer an elliptical, but more likely a highly-inclined spiral. It should be noted that these labels only classify the \emph{projected} appearance; since ellipticals are tri-axial structures, this classification system is very dependent on the orientation angle of any ellipticals which are not perfectly spherical.  

\subsection{Spirals}

The right side of the fork contains the various types of spiral galaxies. These all share the feature of having a flattened disk-shape, and tend to have a spherical bulge of stars in the center with spiral arms extending outward. Spirals whos arms originate from the central bulge follow the top of the fork, labeled ``S'', while those whose arms originate at the ends of a central galactic bar follow the bottom, labeled ``SB''. Both types are further classified based on the relative size of the central bulge and tightness of the arms. Those with large bulges and tighter arms are designated with an ``a'' attached to the spiral symbols, or ``b''-``d'' for decreasing bulge sizes and looser appearance of arms. 

\subsection{Lenticulars/S0s}

Lenticular galaxies are placed at the center of the tuning fork, originally thought to be a transition stage to link the elliptical and spiral types. They exhibit the same overall disk-shape as the spirals, but have a smooth appearance rather than defined arms (which can make them difficult to distinguish from true ellipticals). They may or may not contain a galactic bar, giving them Hubble-type classifications of S0 (unbarred) or S0B (barred). 


Hubble originally referred to the galaxies toward the left and right on the fork as ``early'' or ``late''-type, respectively, simply for convenience in describing their relative positions on the sequence. While it is noted in his 1926 paper that any temporal connotation should be disregarded, the terms remain misleading in that it is now well-known that the early types tend to have older stellar populations, and late-types tend to be very young in their evolution. Nevertheless, ``early-type'' and ``late-type'' are still today used interchangeably when referring to ellipticals/S0s and disks. 

\section{Morphology as a tracer of galaxy evolution}

The previous section described the most common morphological types of galaxies observed in the Universe. At this point it may be relevant to question, why are there different types at all? Do the different shapes exhibit different evolutionary pathways, or is the snapshot we see of the distributions simply showing different stages of a track that all galaxies eventually follow? The answers to these questions aren't fully known; however, examining the relationships between the different morphological types and their dynamics can provide strong insights to the full picture. This section will provide some examples of well-known links between morphology and galaxies' evolutionary histories.

\subsection{Color-Morphology Bimodality}

The color of a galaxy is a strong indicator of its recent star formation history. In general, photometrically blue galaxies are in the process of forming new stars, emitting high energy blue light that is detected abundantly in short-wavelength filters. In contrast, galaxies which have ceased forming stars sometime in the past contribute most of their flux to long-wavelength filters, resulting in redder colors. Perhaps surprisingly, there is also a strong correlation between the color of a galaxy and its morphology. The majority of galaxies ($\sim80\%$) have been shown out to $z\sim1$ to follow this relationship: blue galaxies tend to be late-type spirals, and red galaxies tend to be early-type/elliptical \citep{Tully1982,Strateva2001,Baldry2004,Conselice2006,Martin2007,Mignoli2009}. An example is shown in Figure~\ref{fig:cmd}. The vertical axis tracks the $u-r$ color, such that higher values are ``redder'' and smaller values are ``bluer''. The horizontal axis tracks the absolute magnitude: more luminous galaxies towards the left, and dimmer towards the right. Bluer galaxies tend to have more featured morphologies; spiral arms appear more flocculant and clumps of star formation are apparent, generating irregular shapes in the extreme cases. Redder galaxies begin to have a much more smoothed-out and symmettric appearance, encompassing both ellipticals and bulge-dominated lenticulars. Color has long been considered such a strong indicator of morphology that it has been often used as a proxy for morphology when large-scale visual inspection has not been practical \citep{Cooray2005,Lee2007,Salimbeni2008,Simon2009}. This link is strong evidence that the processes which drive both morphology and the cessation of star formation are related in some way \citep{Masters2010,Buta2013}. This topic is explored in greater detail in Chapter~\ref{chap:gzh_red_disks}. 

\begin{figure}
\centering
\includegraphics[width=\textwidth,trim={3cm 3cm 6cm 3cm},clip]{figures/sdss_plot.pdf}
\label{fig:cmd}
\caption{Color vs. Absolute Magntitude Diagram, illustrated using SDSS galaxies. In each color-magnitude bin, a random galaxy was selected meeting the criteria defined by that bin. The bottom-left and upper-right regions contain very few or zero galaxies, a consequence of typical galaxy evolution. As galaxies age and continue forming stars, they build up more stellar mass, increasing their total luminosity. Hence the most luminous galaxies tend to be older and more massive, and unlikely to be dominated by younger stellar populations. This results in a dearth of luminous blue galaxies (bottom left) or faint red galaxies (upper right).  }
\end{figure}

\subsection{Morphology and Stellar populations}

At the most basic level, morphology is simply a tracer of the observed distribution of light in the galaxy, which in turn traces the distribution of stars, gas, and dust. All light is not emitted equally, however: gas and younger, Population I stars will emit more light in optical and UV wavelengths, while older Population II stars emit more strongly in the infrared. Since these populations may have very different light distributions, there is inherenty some dependence on morphology with the wavelengths within which it's observed. 

Morphologies observed in optical bands are sensitive to pockets of star-formation regions, but other features can be obscured due to dust extinction, particularly those comprised of older stellar populations (such as bars); this can give galaxies an overall ``patchy'' appearance. In contrast, they appear smoother in the near-IR, where the effects of dust extinction are reduced and the older stellar populations dominate. An interesting effect occurs as the observation wavelength moves into the mid-IR: here, dust tends to re-radiate the light absorbed from star-formation regions, re-creating the appearance of the optical-band morphology. There has been debate as to whether the optical and near-IR morphologies are de-coupled to the extent that two classification schemes, one for each wavelength range, is justified (e.g. \citet{Block1999}). Chapter~\ref{chap:ukidss} will examine the optical and near-IR morphologies measured by Galaxy Zoo to add to this debate, as well as investigate whether bars are in fact easier to identify in the near-IR.

\begin{figure}
\centering
\includegraphics[width=\textwidth]{figures/ButaM51.png}
\caption{Credit: \citet{Buta2013}, Figure 2.51. Spiral galaxy M51 observed in optical B-band (left), near-IR (middle), and mid-IR (right). Visible in the B-band image are patchy regions of star-formation and dust lanes, which become invisible in the near-IR, giving an overall smoother appearance. The mid-IR image shows regions where the dust re-radiates light absorbed from star-forming regions, giving a similar appearance to the optical image. }
\label{fig:buta51}
\end{figure} 

\subsection{Morphology and Environment}

A galaxy's environment can also be a predictor of its morphology. The morphology-density relationship, first quantified by \citet{Dressler1980}, observes an abundance of elliptical/ early-type morphologies in denser environments \citep{deSouza1982,Postman1984}. Since the merger rate correlates with environment density, it could be suggested that early-types are often the by-products of mergers, as opposed to a stage of isolated secular evolution. 

There is also evidence of an environmental impact on morphology even in the absence of direct merging. For example, ram pressure \citep{Gunn1972} exerted by the local intracluster medium can severly distort the gas distribution in a galaxy, resulting in asymmetries in the disk (ex. NGC 4402; see also Chapter~\ref{chap:gzh_red_disks}).  

\subsection{Bars}
\citet{Buta2013} describes barred galaxies as ``the ultimate in galaxy morphology.'' His reasoning is simple: just by observing an image of a bar, it is easy to identify it as a major perturbation in an otherwise stable system. There is a great deal of truth in this; such a disruption will no doubt have significant effects on the fate of its host galaxy. In this way, bars are arguably one of the most important structural features that can shape a galaxy's evolution. 

A key feature of bars is their ability to drive gas from the outer regions of the galaxy to the center \citep{Athanassoula1992,Friedli1993,Sellwood1993,Shlosman1989,Ann2005}, which can affect the galaxy's evolution in numerous ways. One such consequence is the formation of a pseudo-bulge \citep{Kormendy2004,Sheth2005}. While this is seen in simulations, this theory is difficult to confirm observationally, as the bar may or not be destroyed by this process \citep{Athanassoula2005}, causing difficulty in identifying a correlation between populations of galaxies with both bars and bulges. 

An increased inflow of gas to the center may also increase central star-formation. Several studies have reported an increase in star-formation rates in the central region of barred galaxies vs. their unbarred counterparts \citep{Hawarden1986,Ho1997}, although this may only be true for strong bars. \citet{Martinet1997} and \citet{Zhou2014} find low rates of star-formation in galaxies with weak bars, suggesting they are unable to trigger significant star formation. Strong bars, however, show both the highest and lowest rates of star-formation. \citep{Sheth2005} found a significant portion of barred galaxies with no molecular gas detected in the nuclear region, which may suggest that for these galaxies, the bar has already driven most of the gas to the nuclear region, where it was consumed by star-formation. Bars, then, seem to play two important roles in the star formation history of their host galaxies - both by increasing star formation, and subsequently driving the quenching process.

Bars also may be one of the mechanisms which enables the fueling of an active galactic nucleus (AGN), whose evolutions are believed to be strongly linked to that of their host galaxies \citep{Schawinski2007, Schawinski2010, Antonini2015, Yang2017, Zubovas2017} (and \citet{Heckman2014} for a comprehensive review). The requirements for onset of accretion onto the central SMBH are still unclear, but \citet{Moles1995} argues that non-axisymmetric components of the gravitational potential may be a necessary condition; a requirement which bars easily satisfy. While simulations have shown bars to provide the necessary inflow to ultimately fuel an AGN \citep{Athanassoula1992,Friedli1993}, observations have shown mixed results. Many have found an excess of AGN in barred samples of galaxies \citep{Knapen2000,Oh2012}, while others find no difference \citep{Ho1997,Mulchaey1997,Cheung2015}. A discussion of the discrepencies between these results, along with my own investigation of this topic, is the subject of Chapter~\ref{chap:baragn}. 

The examples listed are only a few of the well-known relationships between the evolution of galaxies and their morphologies. There is little doubt amongst astronomers that morphology and galactic evolution are linked; however, as evident in these examples, some links are still inconclusive and the research of these relationships is still ongoing. Results are becoming more defined now, as methods to classify galaxies according to their morphologies are consantly improving. Some of the results listed from previous decades suffered from low-sample statistics, where it was only feasible to visually classify handfuls of galaxies in a single study. Today, more robust methods are able to categorize galaxies morphologically in a fraction of the time once required. The next section will explore the evolution of classification methods used to obtain galaxy morphologies for such studies. 


\section{Methods for morphological classification}

Historically, most methods of morphological classification been done by visual inspection of small samples of images (e.g. \citet{Hubble1926,Sandage1961,DeVaucouleurs1963,Block1994,Eskridge2002,Buta2010}), by either a single person or handful of experts. This method is becoming obsolete as we enter a new era of large data, with surveys such as the \href{www.sdss.org}{Sloan Digital Sky Survey (SDSS)} and \href{https://hla.stsci.edu/}{HST-Legacy}, and upcoming \href{https://www.jwst.nasa.gov/}{James Webb Space Telescope} and the \href{https://www.lsst.org/}{Large Synoptic Survey Telescope}, producing high-quality images of hudreds of thousands of galaxies. To date, the largest morphological catalogs created by visual inspection from a small group of experts includes the Nair and Abraham catalog \citep{Nair2010} with $\sim$ 14,000 galaxies, RC3 Catalog \citep{RC31991} with $\sim$ 23,000 galaxies, and MOSES \citep{Schawinski2007} with 50,000 galaxies. Even these catalogs, while successful, do not compare in size to the newly incoming data, and so more powerful and robust efforts are required to obtain morphological information on these scales.


An ideal method for handling the large amounts of data would be an automated classification scheme. Several such algorithms have been developed, with some success \citep{Odewahn2002,Peng2002,Conselice2003,Scarlata2007} by using the stellar light distribution of the galaxy to assign it a morphological class. These approaches tend to be limited to identifying the global morphologies (ie, spiral or elliptical), and lack the precision to accurately identify finer, detailed features (such as bars or the number of spiral arms) (Beck et al. 2017). Further, they tend to incorporate proxies such as color as their input, which are often not accurate as previously noted. Much more promising techniques are currently being tested which incorporate the use of machine-learning algorithms and neural networks \citep{Dieleman2015, Huertas-Company2015},(Beck et al. 2017), but these require massive and accurate training-sets to perform properly. 
One alternative to direct visual classification of morphologies is the use of proxies such as color, mass, surface brightness profile, or some combination of several. Color is commonly used as a proxy because of its mostly-tight relationship global morphology, in that spirals tend to be red and ellipticals tend to be blue. This type of morphological classification will always suffer from a high degree of inaccuracy, as there is no perfect physical measurement that is 100\% correlated with shape. The morphology of a galaxy typically traces the dynamical history, whereas proxies such as color trace stellar growth; these two properties thus reveal different evolutionary histories on possibly very different timescales \citet{Fortson2012}. Last, while there are several proxies which correlate somewhat with the probability of a galaxy being spiral or elliptical, very few could be used to identify finer substructures or more detailed morphological features within the overall shape. 

A best-of-both-worlds approach uses the power of crowdsourcing, which uses the input of tens of thousands of individuals to visually classify galaxies in a fraction of the time achievable by a handful of experts; such a method was developed by Galaxy Zoo, the data from which is used throughout this thesis. The Galaxy Zoo project uses a simple online interface whereby images of galaxies are visually inspected by volunteers, which allows the identification of intricate morphological features to a higher degree of accuracy than computer algorithms today can achieve. Addionally, with a required 40+ independant classifications per galaxy, the resulting classifications carry a greater statistical significance than those generated from one or a few experts. The next chapter will describe how Galaxy Zoo collects data from volunteer citizen scientists, how the data is reduced and debiased, and finally how the data is used to assign morphological classifications to large samples of galaxies.  


% end of chapter






