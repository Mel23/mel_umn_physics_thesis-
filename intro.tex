%%%%%%%%%%%%%%%%%%%%%%%%%%%%%%%%
% intro.tex: Introduction to the thesis
%%%%%%%%%%%%%%%%%%%%%%%%%%%%%%%%
\chapter{Introduction}
\label{chap:intro}

The Intro
reminder to include: discussion on wavelength-dependence of morphological classifications. Discuss the transition of star formation coming through in optical, disappearing in near-IR, then reappearing in mid-IR (Galaxy Morphology Buta 2013 and briefly at end of Buta 2010).  

\section{Methods for morphological classification}

Historically, most methods of morphological classification been done by visual inspection of small samples of images (e.g. \citet{Hubble1926,Sandage1961,DeVaucouleurs1963,Block1994,Eskridge2002,Buta2010}), by either a single person or handful of experts. This method is becoming obsolete as we enter a new era of large data, with recent surveys such as SDSS and HST-Legacy, and upcoming JWST and LSST, producing high-quality images of hudreds of thousands of galaxies. To date, the largest morphological catalogs created by visual inspection from a small group of experts includes the Nair and Abraham catalog \citep{Nair2010} with $\sim$ 14,000 galaxies, RC3 Catalog \citep{RC31991} with $\sim$ 23,000 galaxies, and MOSES \citep{Schawinski2007} with 50,000 galaxies. Even these catalogs, while successful, do not compare in size to the newly incoming data, and so more powerful and robust efforts are required to obtain morphological information on these scales.

One alternative to direct visual classification of morphologies is the use of proxies such as color, mass, surface brightness profile, or some combination of several. Color is commonly used as a proxy because of its mostly-tight relationship global morphology, in that spirals tend to be red and ellipticals tend to be blue. This type of morphological classification will always suffer from a high degree of inaccuracy, as there is no perfect physical measurement that is 100\% correlated with shape. The morphology of a galaxy traces the dynamical history, where proxies such as color trace stellar growth; these two properties thus reveal different evolutionary histories on possibly very different timescales \citet{Fortson2011}. Last, while there are several proxies which correlate somewhat with the probability of a galaxy being spiral or elliptical, very few could be used to identify finer substructures or more detailed morphological features within the overall shape. 

An ideal method for handling the large amounts of data would be an automated classification scheme. Several such algorithms have been developed, with some success \citep{Odewahn2002,Peng2002,Conselice2003} by using the stellar light distribution of the galaxy to assign it a morphological class. These approaches tend to be limited to identifying the global morphologies (ie, spiral or elliptical), and lack the precision to accurately identify finer, detailed features (such as bars or the number of spiral arms) (Beck et al. 2017). Further, they tend to incorporate proxies such as color as their input, which are often not accurate as previously noted. Much more promising techniques are currently being tested which incorporate the use of machine-learning algorithms and neural networks \citep{Dieleman2015, Huertas-Company2015},(Beck et al.2017), but these require massive and accurate training-sets to perform properly. 

A best-of-both-worlds approach uses the power of crowdsourcing, which uses the input of thousands of individuals to visually classify galaxies in a fraction of the time achievable by a handful of experts; such a method was developed by Galaxy Zoo, the data from which is used throughout this thesis. The Galaxy Zoo project uses a simple online interface whereby images of galaxies are visually inspected by volunteers, which allows the identification of intricate morphological features to a higher degree of accuracy than computer algorithms today can achieve. Addionally, with a required 40+ independant classifications per galaxy, the resulting classifications carry a greater statistical significance than those generated from one or a few experts. The next chapter will describe how Galaxy Zoo collects data from volunteer citizen scientists, how the data is reduced and debiased, and finally how the data is used to assign morphological classifications to large samples of galaxies.  


% end of chapter






