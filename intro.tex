%%%%%%%%%%%%%%%%%%%%%%%%%%%%%%%%
% intro.tex: Introduction to the thesis
%%%%%%%%%%%%%%%%%%%%%%%%%%%%%%%%
\chapter{Introduction}
\label{chap:intro}

The Intro
reminder to include: discussion on wavelength-dependence of morphological classifications. Discuss the transition of star formation coming through in optical, disappearing in near-IR, then reappearing in mid-IR (Galaxy Morphology Buta 2013 and briefly at end of Buta 2010). 

Super inspiring first paragraph. Segue: now we'll take a moment to breifly introduce/describe/define important morphologial types. 

\section{Morphological Categorization of Galaxies}

The oldest and most well-known system which categorizes galaxies based on their structure was developed by Edwin Hubble, commonly known as the ``Hubble Tuning-Fork'' \citep{Hubble1926}. Using a small sample of photometric images of nearby galaxies, Hubble identified two fundamental morphological classes: spirals, which exhibited well-defined disk structure and clear spiral arms, and ellipticals, whose light distributions were smoothed over a roughly spherical shape. Only 3\% of the sample had structures which deviated from these two categories, showing no evidence of rotational symmetry about a dominating nucleus; these were grouped together and labeled ``Irregular''. Although Hubble's system was originally based on a mere 400 galaxies, the classifications are still valid for describing the morphologies of the millions of galaxies identifiable today (albeit with some modications, ie. DeVaucouleur's revised system \citep{DeVaucouleurs1963}).

An example of Hubble's Tuning Fork is shown in Figure \ref{fig:tuningfork}. The classifications defined on the Tuning Fork are as follows:

\begin{figure}
\centering
\includegraphics[width=\textwidth]{figures/TuningForkKaren.jpg}
\label{fig:tuningfork}
\caption{The Hubble Tuning fork with gri-composite SDSS images as examples of the various types. Credit: Karen Masters and The Sloan Digital Sky Survey (SDSS) Collaboration.}
\end{figure}

\subsection{Ellipticals}

The left side of the tuning fork contains elliptical galaxies, labeled ``E''. These were originally identified as circular through flattened ellipses whose luminosity faded smoothly from the center to ``indefinite edges.'' The only other structural feature evident to subdivide this class were their ellipticities, defined in the traditional way $e=(a-b)/a$. A number is added to the label that represents the ellipticity, with the decimal omitted, whereby E0 would represent a purely spherical elliptical ($e=0$), and E7 being the most elongated ($e=0.7$). Hubble assumed that any galaxy with an ellipticity higher than 0.7 was no longer an elliptical, but more likely a highly-inclined spiral. It should be noted that these labels only classify the \emph{projected} appearance; since ellipticals are tri-axial structures, this classification system is very dependent on the orientation angle of any ellipticals which are not perfectly spherical.  

\subsection{Spirals}

The right side of the fork contains the various types of spiral galaxies. These all share the feature of having a flattened disk-shape, and tend to have a spherical bulge of stars in the center with spiral arms extending outward. Spirals whos arms originate from the central bulge follow the top of the fork, labeled ``S'', while those whose arms originate at the ends of a central galactic bar follow the bottom, labeled ``SB''. Both types are further classified based on the relative size of the central bulge and tightness of the arms. Those with large bulges and tighter arms are designated with an ``a'' attached to the spiral symbols, or ``b''-``d'' for decreasing bulge sizes and looser appearance of arms. 

\subsection{Lenticulars/S0s}

Lenticular galaxies are placed at the center of the tuning fork, originally thought to be a transition stage to link the elliptical and spiral types. They exhibit the same overall disk-shape as the spirals, but have a smooth appearance rather than defined arms (which can make them difficult to distinguish from true ellipticals). They may or may not contain a galactic bar, giving them Hubble-type classifications of S0 (unbarred) or S0B (barred). 


Hubble originally referred to the galaxies toward the left and right on the fork as ``early'' or ``late''-type, respectively, simply for convenience in describing their relative positions on the sequence. While it is noted in his 1926 paper that any temporal connotation should be disregarded, the terms remain misleading in that it is now well-known that the early types tend to have older stellar populations, and late-types tend to be very young in their evolution. Nevertheless, ``early-type'' and ``late-type'' are still today used interchangeably when referring to ellipticals/S0s and disks. 

\section{Morphology as a tracer of galaxies' dynamical histories}

The previous section described the most common morphological types of galaxies observed in the Universe. At this point it may be relevant to question, why are there different types at all? Do the different shapes exhibit different evolutionary pathways, or is the snapshot we see of the distributions simply showing different stages of a track that all galaxies eventually follow? The answers to these questions aren't fully known; however, examining the relationships between the different morphological types and their dynamics can provide strong insights to the full picture. This section will provide some examples of well-known links between morphology and galaxies' evolutionary histories. 
 
\section{Methods for morphological classification}

Historically, most methods of morphological classification been done by visual inspection of small samples of images (e.g. \citet{Hubble1926,Sandage1961,DeVaucouleurs1963,Block1994,Eskridge2002,Buta2010}), by either a single person or handful of experts. This method is becoming obsolete as we enter a new era of large data, with recent surveys such as SDSS and HST-Legacy, and upcoming JWST and LSST, producing high-quality images of hudreds of thousands of galaxies. To date, the largest morphological catalogs created by visual inspection from a small group of experts includes the Nair and Abraham catalog \citep{Nair2010} with $\sim$ 14,000 galaxies, RC3 Catalog \citep{RC31991} with $\sim$ 23,000 galaxies, and MOSES \citep{Schawinski2007} with 50,000 galaxies. Even these catalogs, while successful, do not compare in size to the newly incoming data, and so more powerful and robust efforts are required to obtain morphological information on these scales.

One alternative to direct visual classification of morphologies is the use of proxies such as color, mass, surface brightness profile, or some combination of several. Color is commonly used as a proxy because of its mostly-tight relationship global morphology, in that spirals tend to be red and ellipticals tend to be blue. This type of morphological classification will always suffer from a high degree of inaccuracy, as there is no perfect physical measurement that is 100\% correlated with shape. The morphology of a galaxy traces the dynamical history, where proxies such as color trace stellar growth; these two properties thus reveal different evolutionary histories on possibly very different timescales \citet{Fortson2011}. Last, while there are several proxies which correlate somewhat with the probability of a galaxy being spiral or elliptical, very few could be used to identify finer substructures or more detailed morphological features within the overall shape. 

An ideal method for handling the large amounts of data would be an automated classification scheme. Several such algorithms have been developed, with some success \citep{Odewahn2002,Peng2002,Conselice2003} by using the stellar light distribution of the galaxy to assign it a morphological class. These approaches tend to be limited to identifying the global morphologies (ie, spiral or elliptical), and lack the precision to accurately identify finer, detailed features (such as bars or the number of spiral arms) (Beck et al. 2017). Further, they tend to incorporate proxies such as color as their input, which are often not accurate as previously noted. Much more promising techniques are currently being tested which incorporate the use of machine-learning algorithms and neural networks \citep{Dieleman2015, Huertas-Company2015},(Beck et al.2017), but these require massive and accurate training-sets to perform properly. 

A best-of-both-worlds approach uses the power of crowdsourcing, which uses the input of thousands of individuals to visually classify galaxies in a fraction of the time achievable by a handful of experts; such a method was developed by Galaxy Zoo, the data from which is used throughout this thesis. The Galaxy Zoo project uses a simple online interface whereby images of galaxies are visually inspected by volunteers, which allows the identification of intricate morphological features to a higher degree of accuracy than computer algorithms today can achieve. Addionally, with a required 40+ independant classifications per galaxy, the resulting classifications carry a greater statistical significance than those generated from one or a few experts. The next chapter will describe how Galaxy Zoo collects data from volunteer citizen scientists, how the data is reduced and debiased, and finally how the data is used to assign morphological classifications to large samples of galaxies.  


% end of chapter






