
% CHAPTER:  1
% (Note: cannot have a footnote on a word within the \chapter{} construct, it does not work)
\chapter{UKIDSS}
\label{chap:ukidss}

\section{Intro: wavelength dependence on morphology: optical and IR}

Historically, visual morphological classification of galaxies has been conducted on optical images. Blue B-band images were the primary source dating back to Hubble's classic tuning-fork classification scheme \citep{Hubble1926} and in the subsequent modifications by \citet{Sandage1961} and \citet{DeVaucouleurs1963}. The more recent and larger morphological catalogs also derive their classifications from optical images, either single-band (\citet{RC31991} (B-band), \citet{Scarlata2007} (ACS I-F814W), \citet{Fukugita2007} and \citet{Nair2010} (SDSS g-band)) or color-composite (\citet{Lintott2008}, \citet{Willett2013} (SDSS-gri)). 

In the optical regime, the flux is dominated by young, hot stars; this results in an emphasis of spiral structure in the images, but they tend to have patchy appearances due to the abundance of star-formation regions in the arms. Optical images also are impacted by extinction due to dust, which can obscure features that tend to be composed of older stellar components (such as bars and bulges). Longer wavelenths are free of these effects, making them ideal for revealing the underlying ``stellar backbone'' of galaxies.

It is possible, then, to consider two morphologically distinct components of a galaxy: a gas-dominated Population I disk, and a star-dominated Population II disk. The Population I disk is most easily seen in the optical, revealing HII regions, cold HI gas, and emission from young OB stars; these regions will tend to highlight spiral structure. The Population II disk, on the other hand, traces the underlying mass distribution; consisting of the old, cooler stellar population, it is more easily seen at longer wavelengths. \citet{Block1999} even suggests that two separate classifications schemes should be required for all galaxies; one for the Population I disk, which can be probed in optical and ultraviolet images, and a Population II disk, for which longer wavelength images, free of dust extinction, would be required. 

The extent to which the morphologies of the younger and older stellar populations are decoupled, however, is not yet clear. Early studies which directly compared optical and near-IR images found very significant differences between the two morphologies \citep{Hackwell1983,Thronson1989,Block1991}. \citet{Block1999} goes as far as to sugget that there is no correlation between the two, and that the optically-definied Hubble tuning fork ``does not constrain the morphology of the old steallar Population II disks.'' However, all of the aforementioned studies only compared morphologies of a single galaxy (or in the case of \citet{Block1999}, a handful), so these conclusions cannot be applied generally.

Next: Go into detail of recent studies with large samples (Buta 2010, Eskridge 2002, Elmegreen 2011,any others??); most of these I think find strong correlation in Hubble type between IR and optical.

Discuss: near-IR vs mid-IR results

Then: focus on bars. Note early results of Hackwell1983 and Thronson1989, both found previously invisible bars. More recent: Mendendez-Delmestra2007, find more!
!



\section{UKIDSS sample}

The UKIDSS sample is comprised of 71,052 infrared images of galaxies which had been previously optically classified in GZ2. The images were taken with the United Kingdom Infrared Telescope (UKIRT) as part of the UKIRT Infrared Deep Sky Survey (UKIDSS; \citet{Lawrence2007,Warren2007}. The Large Area Survey (LAS) portion of UKIDSS covered the SDSS observations at high Galactic altitudes, allowing for full YZJHK coverage.  

Morphological classifications for the UKIDSS sample were obtained via Galaxy Zoo, where users were shown YJK color-composite images. The classification tree used was identical to that in GZ2, allowing a direct comparison of morphologies using the same vote fractions. Raw votes were counted and weighted by user consistency in the same manner as the GZ2 sample (details of this process in \citet{Willett2013}).  
