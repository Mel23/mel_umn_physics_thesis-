%%%%%%%%%%%%%%%%%%%%%%%%%%%%%%%%%%%%%%%%%%%%%%%%%%%%%%%%%%%%%
% abstract.tex: Abstract
%%%%%%%%%%%%%%%%%%%%%%%%%%%%%%%%%%%%%%%%%%%%%%%%%%%%%%%%%%%%%

%Galaxies are cool and have pretty colors sometimes. Can I has degree now? Thx :)

%\bigskip 

Galaxy morphology is one of the primary keys to understanding a galaxy's evolutionary history. External mechanisms (environment/clustering, mergers) have a strong impact on formative evolution of the major galactic components (disk, bulge, Hubble type), while internal instabilities created by bars, spiral arms, or other substructures drive secular evolution via the rearrangement of material within the disk. This thesis will explore several ways in which morphology may impact the dynamics and evolution of a galaxy using visual classifications from several Galaxy Zoo projects. Section 1 will focus on the present morphology of galaxies in the local Universe ($z<0.2$) using data from Galaxy Zoo 2 and Galaxy Zoo UKIDSS. Section 2 will examine populations of morphologies at various lookback times, from $z=0$ out to $z=1$ using data from Galaxy Zoo Hubble.


%%GZ2 bar/agn study
We first explore the impact of bars in disc galaxies on channeling gas from the outer regions of the disk to the inner few kpc necessary to fuel an active galactice nucleus (AGN). Using a sample of 19,756 disk galaxies at $0.01 < z < 0.05$ imaged by the Sloan Digital Sky Survey and morphologically classified by Galaxy Zoo 2, the difference in AGN fraction in barred and unbarred disks was measured. A weak, but statistically significant, effect was found in that the population of AGN hosts exhibited a 16.0\% increase in bar fraction as compared to their unbarred counterparts at fixed mass and color. These results are consistent with a cosmological model in which bar-driving fueling contributes to the fueling of growing black holes, but other dynamical mechanisms must also play a significant role. 

%%study/results of GZ UKIDSS
We study the wavelength dependence on morphology by comparing the optical morphological classifications from GZ2 to classifications done on infrared images in GZ:UKIDSS. We find some cool result. [to be continued]

% study/results of GZH science
We examine more directly the morphological changes in galaxy populations as a function of their age using classifications from Galaxy Zoo: Hubble. A sample of XX,XXX disc galaxies from the COSMOS field at $0<z<1$ were identified as active or passive using a NUV-r / r -J diagnostic with rest-frame colors from the UltraVISTA catalog. We find that the fraction of disks that are passive increases/decreases from X.X\% at $z=1$ to X.X\% at $z=0$. We intepret this result as [something having to do with the transformation of disk to elliptical, depending on result]. Addionally, we emphasize the challenges of visual classification that are particular to galaxies at high redshift. We present a correction technique to address these biases using simulated images of nearby SDSS galaxies which were artificially redshifted using the FERENGI code and classified in GZH.  








